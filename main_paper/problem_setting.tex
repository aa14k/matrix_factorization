\section{Preliminaries}
\textbf{(HARALD/DAWEN)}
% As is common in many papers on recommender systems with implicit feedback data (\citet{hu2008collaborative,liang2018variational,steck2019embarrassingly}), we assume access to a matrix (typically sparse and binary (\citet{ning2011slim})) of user-by-item interactions $X \in \mathbb{R}^{|\mathcal{U}|\times|\mathcal{I}|}$ where $\mathcal{U} = \{1,2,\cdots,U\}$ and $\mathcal{I} = \{1,2,\cdots,I\}$ denotes the set of users and items respectively. If $X_{ui}$ is positive, then we say say a user $u$ has interacted with an item $i$. If $X_{ui}$ is zero, then we say a user $u$ has no interaction with item $i$.  

% In this work, we ``binarize'' the user-by-item interaction matrix so that all the positive entries of $X$ take value $1$ and the zero entries of $X$ remain unchanged. This is done to facilitate the use of standard ranking metrics such as the truncated normalized discounted cumulative gain (NDCG$@R$) (\citet{jarvelin2000ir}) and Recall$@R$, which we will definite shortly. The goal of many collaborative filtering-based approaches, such as matrix factorization, is to learn an item similarity (or embedding) matrix, $B$, that gives a predicted score for an item $i \in \mathcal{I}$ given a user $u \in \mathcal{U}$. Said formally, we aim to learn a matrix $B \in \mathbb{R}^{d \times |\mathcal{I}|}$ such that given a user embedding $\tilde{X}_u \in \mathbb{R}^d$, the predicted score for the user-item pair is the inner product between the given user embedding and the $i$-th column of $B$, 
% \begin{equation*}
%     S_{ui} = \langle\tilde{X}_u, B_{\cdot,i}\rangle\,.
% \end{equation*} 
% In this work, we consider simplified methods based on \textit{linear} autoencoders trained via regularized least squares, as was done in \citet{NEURIPS2020_e33d974a}. This is done to facilitate analytic insights into the behaviour of such methods and to better understand the behavior of deep nonlinear models (\citet{NIPS2016_f2fc9902,pretorius2018learning,advani2020high,NEURIPS2020_e33d974a}). The linear autoencoder is given by the item similarity matrix $B$. If the user embedding $\tilde{X} = X$  