%%
%% This is file `sample-sigconf.tex',
%% generated with the docstrip utility.
%%
%% The original source files were:
%%
%% samples.dtx  (with options: `all,proceedings,bibtex,sigconf')
%% 
%% IMPORTANT NOTICE:
%% 
%% For the copyright see the source file.
%% 
%% Any modified versions of this file must be renamed
%% with new filenames distinct from sample-sigconf.tex.
%% 
%% For distribution of the original source see the terms
%% for copying and modification in the file samples.dtx.
%% 
%% This generated file may be distributed as long as the
%% original source files, as listed above, are part of the
%% same distribution. (The sources need not necessarily be
%% in the same archive or directory.)
%%
%%
%% Commands for TeXCount
%TC:macro \cite [option:text,text]
%TC:macro \citep [option:text,text]
%TC:macro \citet [option:text,text]
%TC:envir table 0 1
%TC:envir table* 0 1
%TC:envir tabular [ignore] word
%TC:envir displaymath 0 word
%TC:envir math 0 word
%TC:envir comment 0 0
%%
%%
%% The first command in your LaTeX source must be the \documentclass
%% command.
%%
%% For submission and review of your manuscript please change the
%% command to \documentclass[manuscript, screen, review]{acmart}.
%%
%% When submitting camera ready or to TAPS, please change the command
%% to \documentclass[sigconf]{acmart} or whichever template is required
%% for your publication.
%%
%%
\documentclass[sigconf, anonymous, review]{acmart}

\usepackage{amsmath}
\usepackage{amsthm}
\usepackage{mathtools}
\usepackage{ulem}
\usepackage{enumerate}
\usepackage{xcolor}

\usepackage[shortlabels]{enumitem} 
\usepackage{cleveref} 

\DeclareMathOperator{\ve}{vec}
\DeclareMathOperator{\di}{diagMat}

%%
%% \BibTeX command to typeset BibTeX logo in the docs
\AtBeginDocument{%
  \providecommand\BibTeX{{%
    Bib\TeX}}}

%% Rights management information.  This information is sent to you
%% when you complete the rights form.  These commands have SAMPLE
%% values in them; it is your responsibility as an author to replace
%% the commands and values with those provided to you when you
%% complete the rights form.
%\setcopyright{acmlicensed}
%\copyrightyear{2018}
%\acmYear{2018}
%\acmDOI{XXXXXXX.XXXXXXX}

%% These commands are for a PROCEEDINGS abstract or paper.
%\acmConference[Conference acronym 'XX]{Make sure to enter the correct
%  conference title from your rights confirmation emai}{June 03--05,
%  2018}{Woodstock, NY}
%%
%%  Uncomment \acmBooktitle if the title of the proceedings is different
%%  from ``Proceedings of ...''!
%%
%%\acmBooktitle{Woodstock '18: ACM Symposium on Neural Gaze Detection,
%%  June 03--05, 2018, Woodstock, NY}
%\acmISBN{978-1-4503-XXXX-X/18/06}


%%
%% Submission ID.
%% Use this when submitting an article to a sponsored event. You'll
%% receive a unique submission ID from the organizers
%% of the event, and this ID should be used as the parameter to this command.
%%\acmSubmissionID{123-A56-BU3}

%%
%% For managing citations, it is recommended to use bibliography
%% files in BibTeX format.
%%
%% You can then either use BibTeX with the ACM-Reference-Format style,
%% or BibLaTeX with the acmnumeric or acmauthoryear sytles, that include
%% support for advanced citation of software artefact from the
%% biblatex-software package, also separately available on CTAN.
%%
%% Look at the sample-*-biblatex.tex files for templates showcasing
%% the biblatex styles.
%%

%%
%% The majority of ACM publications use numbered citations and
%% references.  The command \citestyle{authoryear} switches to the
%% "author year" style.
%%
%% If you are preparing content for an event
%% sponsored by ACM SIGGRAPH, you must use the "author year" style of
%% citations and references.
%% Uncommenting
%% the next command will enable that style.
%%\citestyle{acmauthoryear}


%%
%% end of the preamble, start of the body of the document source.
\begin{document}

%%
%% The "title" command has an optional parameter,
%% allowing the author to define a "short title" to be used in page headers.
\title{Does weighting improve matrix factorization?}

%%
%% The "author" command and its associated commands are used to define
%% the authors and their affiliations.
%% Of note is the shared affiliation of the first two authors, and the
%% "authornote" and "authornotemark" commands
%% used to denote shared contribution to the research.


%%
%% By default, the full list of authors will be used in the page
%% headers. Often, this list is too long, and will overlap
%% other information printed in the page headers. This command allows
%% the author to define a more concise list
%% of authors' names for this purpose.
%\renewcommand{\shortauthors}{Trovato et al.}

%%
%% The abstract is a short summary of the work to be presented in the
%% article.
\begin{abstract}
Matrix factorization is a widely used approach for top-N recommendations and collaborative filtering. When implemented on implicit feedback data (such as clicks), a common heuristic is to upweight the nonzero labels; some algorithms have been shown to perform better when fit to weighted data. In this work, we systematically study different weighting schemes and algorithms, and somewhat surprisingly conclude that the best methods, as measured by ranking accuracy on publicly available data sets, use unweighted data. In doing so, we provide some theoretical insights into why weighting improves the performance of some methods but may actually hinder the performance of others.
\end{abstract}

%%
%% The code below is generated by the tool at http://dl.acm.org/ccs.cfm.
%% Please copy and paste the code instead of the example below.
%%
% \begin{CCSXML}
% <ccs2012>
%  <concept>
%   <concept_id>00000000.0000000.0000000</concept_id>
%   <concept_desc>Do Not Use This Code, Generate the Correct Terms for Your Paper</concept_desc>
%   <concept_significance>500</concept_significance>
%  </concept>
%  <concept>
%   <concept_id>00000000.00000000.00000000</concept_id>
%   <concept_desc>Do Not Use This Code, Generate the Correct Terms for Your Paper</concept_desc>
%   <concept_significance>300</concept_significance>
%  </concept>
%  <concept>
%   <concept_id>00000000.00000000.00000000</concept_id>
%   <concept_desc>Do Not Use This Code, Generate the Correct Terms for Your Paper</concept_desc>
%   <concept_significance>100</concept_significance>
%  </concept>
%  <concept>
%   <concept_id>00000000.00000000.00000000</concept_id>
%   <concept_desc>Do Not Use This Code, Generate the Correct Terms for Your Paper</concept_desc>
%   <concept_significance>100</concept_significance>
%  </concept>
% </ccs2012>
% \end{CCSXML}

%\ccsdesc[500]{Do Not Use This Code~Generate the Correct Terms for Your Paper}
%\ccsdesc[300]{Do Not Use This Code~Generate the Correct Terms for Your Paper}
%\ccsdesc{Do Not Use This Code~Generate the Correct Terms for Your Paper}
%\ccsdesc[100]{Do Not Use This Code~Generate the Correct Terms for Your Paper}

%%
%% Keywords. The author(s) should pick words that accurately describe
%% the work being presented. Separate the keywords with commas.
\keywords{Recommender System; Collaborative Filtering; Autoencoder; Neighborhood Approach; Weighted Linear Regression; Matrix Factorization}
%% A "teaser" image appears between the author and affiliation
%% information and the body of the document, and typically spans the
%% page.
% \begin{teaserfigure}
%   \includegraphics[width=\textwidth]{sampleteaser}
%   \caption{Seattle Mariners at Spring Training, 2010.}
%   \Description{Enjoying the baseball game from the third-base
%   seats. Ichiro Suzuki preparing to bat.}
%   \label{fig:teaser}
% \end{teaserfigure}

%\received{20 February 2007}
%\received[revised]{12 March 2009}
%\received[accepted]{5 June 2009}

%%
%% This command processes the author and affiliation and title
%% information and builds the first part of the formatted document.
\maketitle

\section{Introduction}
\textbf{(HARALD/DAWEN)}
\section{Preliminaries}
\textbf{(HARALD/DAWEN)}
% As is common in many papers on recommender systems with implicit feedback data (\citet{hu2008collaborative,liang2018variational,steck2019embarrassingly}), we assume access to a matrix (typically sparse and binary (\citet{ning2011slim})) of user-by-item interactions $X \in \mathbb{R}^{|\mathcal{U}|\times|\mathcal{I}|}$ where $\mathcal{U} = \{1,2,\cdots,U\}$ and $\mathcal{I} = \{1,2,\cdots,I\}$ denotes the set of users and items respectively. If $X_{ui}$ is positive, then we say say a user $u$ has interacted with an item $i$. If $X_{ui}$ is zero, then we say a user $u$ has no interaction with item $i$.  

% In this work, we ``binarize'' the user-by-item interaction matrix so that all the positive entries of $X$ take value $1$ and the zero entries of $X$ remain unchanged. This is done to facilitate the use of standard ranking metrics such as the truncated normalized discounted cumulative gain (NDCG$@R$) (\citet{jarvelin2000ir}) and Recall$@R$, which we will definite shortly. The goal of many collaborative filtering-based approaches, such as matrix factorization, is to learn an item similarity (or embedding) matrix, $B$, that gives a predicted score for an item $i \in \mathcal{I}$ given a user $u \in \mathcal{U}$. Said formally, we aim to learn a matrix $B \in \mathbb{R}^{d \times |\mathcal{I}|}$ such that given a user embedding $\tilde{X}_u \in \mathbb{R}^d$, the predicted score for the user-item pair is the inner product between the given user embedding and the $i$-th column of $B$, 
% \begin{equation*}
%     S_{ui} = \langle\tilde{X}_u, B_{\cdot,i}\rangle\,.
% \end{equation*} 
% In this work, we consider simplified methods based on \textit{linear} autoencoders trained via regularized least squares, as was done in \citet{NEURIPS2020_e33d974a}. This is done to facilitate analytic insights into the behaviour of such methods and to better understand the behavior of deep nonlinear models (\citet{NIPS2016_f2fc9902,pretorius2018learning,advani2020high,NEURIPS2020_e33d974a}). The linear autoencoder is given by the item similarity matrix $B$. If the user embedding $\tilde{X} = X$  
\section{(Asymmetrical) Matrix Factorization}
In this section, we introduce and review our objective function for learning the item-similarity matrix $B$. In this work, we learn the weights $B$ via regularized least squares:
\begin{equation}\label{eqn:ease}
    \min_{B} \left\lVert \sqrt{W} \odot (X - XB)\right\rVert_F^2 + \lambda \|B\|_F^2\,,
\end{equation}
where $W \in \mathbb{R}_{\geq 0}^{|\mathcal{U}|\times |\mathcal{I}|}$ is a set of non-negative weights. We also consider two low rank variants of the above model, first with ``dropout'' style regularization
\begin{align}\label{eqn:awmf-dropout}
    \min_{U,V\in\mathbb{R}^{|\mathcal{I}|\times d}} \left\lVert \sqrt{W} \odot (X - XUV^\top)\right\rVert_F^2 + \lambda\|UV^\top\|_F^2\, \quad\text{ and }
\end{align}
and then with ``weight decay'' style regularization.
\begin{align}\label{eqn:awmf-weightdecay}
    \min_{U,V\in\mathbb{R}^{|\mathcal{I}|\times d}} \left\lVert \sqrt{W} \odot (X - XUV^\top)\right\rVert_F^2 + \lambda\|U\|_F^2 + \lambda \|V\|_F^2\,
\end{align}
where $d \leq \mathcal{I}$. The insight that the regularization schemes in \cref{eqn:awmf-dropout,eqn:awmf-weightdecay} can be related to dropout and weight decay is due to \citet{steck2021regularization}. 

The above objectives can minimized via alternating minimization (\citet{cichocki2007regularized}). We mention that both \citet{ye2021global} and \citet{lee2023randomly} show that the global optima of \cref{eqn:awmf-dropout,eqn:awmf-weightdecay} can be found via alternating minimization when both $U,V$ are initialized from a normal distribution, $W$ is a multiple of the ones matrix and $\lambda = 0$. It remains open whether such a result holds when $W$ is not a constant matrix and $\ell_2$ regularization is present. Several further comments are in order:
\begin{itemize}
    \item We choose the square loss (here $\| \cdot \|_F$ denotes the Frobenuis norm) as is standard in the literature when applying matrix factorization in collaborative filtering/top-$N$ recommendations (\citet{hu2008collaborative,ning2011slim,steck2019embarrassingly,NEURIPS2020_e33d974a}). \citet{liang2018variational,depauw24} observe that training with different loss functions such as the logistic loss yield better ranking accuracy over the \textit{unweighted} square loss, as the log-loss reweights the data (\citet{ayoubswitching}). However, as will we will argue shortly, training on unweighted data gives the best performance with the added benefit of lower computational costs. 
    \item We will use weights of the form $W = aX + 1$ for $a \geq 0$ as was first proposed by \citet{hu2008collaborative}. \textbf{NEED AN EXPLANATION FOR THIS!!!!}
    \item We will use various forms of $\ell_2$-regularization. As was observed by \citet{NEURIPS2020_e33d974a,steck2021regularization} and \citet{jin2021towards}, if we let the item similarity matrix be low rank, i.e., $B=UV^\top$, then different regularization schemes can lead to different ranking accuracy. Note that this adds an additional hyperparameter $\lambda$ to be optimized on a held-out validation set. 
\end{itemize}

\subsection{Closed-Form Solution: Unregularized}
In this section, we show that the objective functions for learning the item similarity matrices $B$ and $UV^\top$ in \cref{eqn:ease} and \cref{eqn:awmf-dropout,eqn:awmf-weightdecay} can be solved in closed form when $\lambda = 0$. The choice to first present the result for the unregularized case is done solely for exposition. In the proceeding section we will extend our analysis to hold when $\lambda > 0$. Our result contradicts a claim made by \citet{steck2019collaborative,jin2021towards} where the authors claim that a closed form solution to \cref{eqn:ease} and \cref{eqn:awmf-dropout,eqn:awmf-weightdecay} does not exist when $W$ is a general weighting-matrix. Thus we resolve an open question by giving a closed form solution to the objectives in \cref{eqn:ease,eqn:awmf-dropout,eqn:awmf-weightdecay}. 

In order to derive our closed form solution, we will first review the definition of the Kronecker product. 
\begin{definition}[Kronecker product]
    The Kronecker product of a matrix $A \in \mathbb{R}^{m\times n}$ and $B \in \mathbb{R}^{p\times q}$ is denoted by $A \otimes B$ and is defined to be the block matrix
    \begin{equation*}
        A\otimes B = 
        \begin{bmatrix}
        a_{11}B & a_{12}B & \cdots & a_{1n}B \\
        a_{21}B & a_{22}B & \cdots & a_{2n}B \\
        \vdots  & \vdots  & \ddots & \vdots  \\
        a_{m1}B & a_{m2}B & \cdots & a_{mn}B\
        \end{bmatrix}\,.
    \end{equation*}
    We refer the reader to Chapter 4.2 of \citet{Horn_Johnson_1991} for elementary properties of the Kronecker product which we will employ in deriving our closed form solution. We will also need the following ``vectorization'' lemma to derive our closed-form solution.
    \begin{lemma}[Lemma 4.3.1 of \citet{Horn_Johnson_1991}]\label{lem:vectorization}
        Let matrices $A \in \mathbb{R}^{m\times n},B \in \mathbb{R}^{p\times q}$ and $C \in \mathbb{R}^{m \times q}$ be given and $X \in \mathbb{R}^{n \times p}$ be unknown. The matrix equation
        \begin{equation*}
            AXB = C
        \end{equation*}
        is equivalent to the system of $qm$ equations in $np$ unknowns given by
        \begin{equation*}
            (B^\top \otimes A)\ve(X) = \ve(C)\,,
        \end{equation*}
        that is, $\ve(AXB) = (B^\top \otimes A)\ve(X)$ where $\ve(A)$ is the $mn\times 1$ column vector obtained by stacking the columns of $A$ on top of one another\footnote{$\ve(A)=A.$flatten('F') in numpy}.
    \end{lemma}
\end{definition}
The vectorization lemma has historical been used to reason about and compute solutions to the Sylvester's equation $AX + XB = C$ and Lypanuov equations $AXA^\top - X + B = 0$, which arise natural when studying linear dynamical systems in control theory. Here one can use the vectorization lemma to ``linearize'' these seemingly nonlinear matrix equations. Our closed-form solutions takes inspiration from this application of the vectorization lemma.

In what follows we first consider the case when $\lambda = 0$ and then, in the following section, extend our analysis to the general case when $\lambda >0$. 
The optimization problem in \cref{eqn:ease} can be solved by taking its derivative and setting it to be zero, 
\begin{equation*}
     X^T\left(W \odot(XB - X) \right) = 0\,.
\end{equation*}
Previous attempts at finding a closed form solution failed due to the nature of the Hadamard (element-wise) product $\odot$, which does not allow the order operations to be exchanged. Employing \cref{lem:vectorization} and re-arranging, we can rewrite the above matrix equation as
\begin{align*}\label{eqn:ease-soln}
    \ve(X^\top(W\odot X)) &= \ve\left(X^T\left(W \odot(XB)\right)\right) \\
    &= (I\otimes X^\top)\ve(W\odot(XB)) \\
    &= (I\otimes X^\top)(\ve(W)\odot\ve(XB)) \\
    &= (I \otimes X^\top)\di(\ve(W))\ve(XB) \\
    &= (I\otimes X^\top)\di(\ve(W))(I\otimes X)\ve(B)\,
\end{align*}
where the second and last equality use \cref{lem:vectorization} and the fourth equality uses the fact that for vectors $x,y$, it holds that $x\odot y = \di(x)y$. Re-arranging the above equation gives the following closed form for $B$
\begin{equation}\label{eqn:closed-form-ease}
    \ve(B) = \big(\underbrace{(I\otimes X^\top)\di(\ve(W))(I\otimes X)}_{=H_B}\big)^{-1} \ve(X^\top(W\odot X))\,.
\end{equation}
Finally, we note the while a closed form solution for $B$ was given in terms of $W$ and $X$, this method can be used for finding a closed form solution for any optimization problem of the form,
\begin{equation*}
    \min_{U,V\in\mathbb{R}^{|\mathcal{I}|\times d}} \left\lVert \sqrt{W} \odot (X - XUV^\top)\right\rVert_F^2\,.
\end{equation*}

We now employ a similar calculation for computing the minimizer of \cref{eqn:awmf-dropout,eqn:awmf-weightdecay} when $\lambda = 0$. Since we will employ alternating minimization for solving \cref{eqn:awmf-dropout,eqn:awmf-weightdecay}, we first fix $V$ and compute the closed form solution for $U$ and then repeat a similar calculation to find the closed form solution for $U$. Taking the derivative of expression to the left of the sum in \cref{eqn:awmf-dropout} with respect to $U$ and setting it to be zero gives
\begin{equation*}
    X^\top(W\odot(XUV^\top - X))V = 0\,.
\end{equation*}
Employing \cref{lem:vectorization} and re-arranging the above display gives us
\begin{align*}
    \ve(X^\top(W\odot X)V) &=  X^\top(W\odot(XUV^\top))V \\
    &= (V^\top\otimes X^\top)\ve(W\odot(XUV^\top))\\
    &= (V^\top\otimes X^\top)\ve(W)\odot\ve(XUV^\top)\\
    &= \underbrace{(V^\top\otimes X^\top)\di(\ve(W))(V\otimes X)}_{=H_U}\ve(U)\,.
\end{align*}
Re-arraigning the above equation gives the following closed form of $U$
\begin{align}\label{eqn:awmfU-soln}
    \ve(U)
    = \big(\underbrace{(V^\top\otimes X^\top)\bar{W}(V\otimes X)}_{=H_V}\big)^{-1} \ve(X^\top(W\odot X)V)\,,
\end{align}
where $\bar{W} = \di(\ve(W))$
Repeating the same calculations for $V$ we get that
\begin{equation}\label{eqn:awmfV-soln}
    \ve(V) = \big((XU^\top \otimes I)\bar{W}(XU\otimes I)\big)^{-1} \ve((W^\top\odot X^\top)XU)
\end{equation}
where $\bar{W} = \di(\ve(W))$. In the next section, we will extend our analysis to hold for any $\lambda > 0$. 
\subsection{Closed-Form Solution: Regularized}
In this section, we show that the regularized objective functions for learning $B$, $U$ and $V$ in \cref{eqn:ease,eqn:awmf-dropout,eqn:awmf-weightdecay} can be solved in closed form.

\subsection{Computational Considerations}
The matrix $B$ can be computed without storing $H$ or computing the Kronecker products if given the singular value decomposition (SVD) of the matrix $X$, i.e., $X = Y\Sigma Z^\top$ where $Y,Z$ are orthonormal matrices and $\Sigma$ is a diagonal matrix. Let $A^+$ denote the Moore-Penrose inverse of $A$ and then it holds that that $(ABC)^+ = C^+B^+A^+$ and if $A$ is invertible then $A^+ = A^{-1}$ (see Section 6C of \citet{axler}). Then given the SVD of $X$ and letting $\bar W = \di(\ve(W))$, 
\begin{align*}
    H^{-1}=H^{+} &= (I\otimes X)^+ \bar W^+ (I\otimes X^\top)^+ \\
    &= (I\otimes Y\Sigma Z^\top)^+ \bar W^+ (I\otimes Z\Sigma Y^\top)^+ \\
    &= (I^+\otimes Z\Sigma^+ Y^\top) \bar W^+ (I^+\otimes Y\Sigma^+ Z^\top) \\
    &= (I\otimes Z\Sigma^+ Y^\top) \bar W^{-1} (I^{-1}\otimes Y\Sigma^+ Z^\top)
\end{align*}
where the third equality follows from the definitions of the Kronecker product and the Moore-Penrose inverse (e.g. Exercise 10.25 \citet{Abadir_Magnus_2005}) and the final equality that $Y,Z$ are orthonormal, i.e., $Y^+ = Y^{-1} = Y^\top$ and $Z^+ = Z^{-1}=Z^\top$. The Moore-Penrose inverse of the diagonal matrix $\Sigma$ is computed by simply inverting all its nonzero entries. Again employing \cref{lem:vectorization}, we have t


\section{Numerical Experiments}
\textbf{(ALEX)}
\section{Related Works}
\textbf{(HARALD/DAWEN)}
\section{Conclusions}
\textbf{(HARALD/DAWEN/ALEX)}
\bibliographystyle{ACM-Reference-Format}
\bibliography{sample-base}

\appendix
\end{document}
\endinput
%%
%% End of file `sample-sigconf.tex'.
